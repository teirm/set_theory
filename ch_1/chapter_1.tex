\documentclass[../base/set_theory.tex]{subfiles}

\begin{document}
\chapter{Classes and Sets}
\section{Historical Introduction and Paradox}
Field began with Cantor and Dedekind with naive set theory.
\begin{itemize}
    \item  Semantic Paradoxes: contradiction arising from how
           statements are made in a system.
    \item  Logical Paradoxes: contradiction arising from structure
           of the system.
\end{itemize}
\section{Russel's Paradox}
Let $S$ be the set of all sets that do not contain themselves.
Is $S \in S$?  If $S \in S$, then by definition of $S$, $S\notin
S$.  If $S \notin S$, then by defintion of $S$, $S \in S$.
\section{Berry's Paradox}
See Page 4 of text.
\section{Zermelo's System}
This system has one primitive notion: $\in$.  By restricting naieve
set theory, we can avoid the lgoical and semantic paradoxes.\\
\\
For example, let $S(x)$ be a condition on $x$.  Can't form the set 
of all $x$ which satisfy $S(x)$, but if $A$ is a given set we can
form the set of all $x$ in $A$ which satisfy $S(x)$.\\
\\
Essentialy, a propery $S(x)$ can't be used to form a set, but it 
can instead be used to \textit{select} elements out of a known set.
This is known as the \textbf{Axiom of Selection}.  This avoids the
logical paradoxes by preventing sets that are ``too large''.\\
\\
The semantic paradoxes are avoided by restricting the statements 
in the system to only those made from a formal language.
\section{Basic Logic}
A \textbf{sentence} is a statement that is either true or false --
\textbf{unambiguously}.\\\\
\break
Given a sentence $\mathcal{P}$, $\neg \mathcal{P}$ is the negation of
$\mathcal{P}$.\\
\begin{center}
    \begin{tabular}{c|c}
    $\mathcal{P}$ & $\neg \mathcal{P}$ \\ \hline
    T & F \\
    F & T \\
    \end{tabular}
    \\[10pt]
    \caption{Truth Table for $\neg \mathcal{P}$}
\end{center}
\\\\
Given sentences $\mathcal{P}$ and $\mathcal{Q}$, the conjunction of
$\mathcal{P}\land \mathcal{Q}$ is true if $\mathcal{P}$ and $\mathcal{Q}$
are both true.\\
\begin{center}
    \begin{tabular}{c|c|c}
    $\mathcal{P}$ & $\mathcal{Q}$ & $\mathcal{P}\land \mathcal{Q}$ \\ \hline
    T & T & T \\ 
    T & F & F \\ 
    F & T & F \\  
    F & F & F \\
    \end{tabular}
    \\[10pt]
    \caption{Truth Table for $\mathcal{P}\land \mathcal{Q}$}
\end{center}
\\\\
The disjuntion of $\mathcal{P}$ and $\mathcal{Q}$ is defined as true if
$\mathcal{P}$ or $\mathcal{Q}$ are true.\\
\begin{center}
    \begin{tabular}{c|c|c}
    $\mathcal{P}$ & $\mathcal{Q}$ & $\mathcal{P}\lor \mathcal{Q}$ \\ \hline
    T & T & T \\ 
    T & F & T \\ 
    F & T & T \\  
    F & F & F \\
    \end{tabular}
    \\[10pt]
    \caption{Truth Table for $\mathcal{P}\lor \mathcal{Q}$}
\end{center}
\\\\
Implication is given by $\mathcal{P}\to \mathcal{Q}$ is especially important.
Formally, $\mathcal{P}\to \mathcal{Q}$ is true, except if $\mathcal{P}$ is true 
and $\mathcal{Q}$ is false.\\
\begin{center}
    \begin{tabular}{c|c|c}
    $\mathcal{P}$ & $\mathcal{Q}$ & $\mathcal{P}\to \mathcal{Q}$ \\ \hline
    T & T & T \\ 
    T & F & F \\ 
    F & T & T \\  
    F & F & T \\
    \end{tabular}
    \\[10pt]
    \caption{Truth Table for $\mathcal{P}\to \mathcal{Q}$}
\end{center}
\\\\
\begin{customthm}{1.5}
    For all sentences $\mathcal{P}$ and $\mathcal{Q}$, the following are true.
    \begin{center}
    \begin{multicols}{2}
    \begin{itemize}
        \item $\mathcal{P}\to \mathcal{P}\lor \mathcal{Q}$
        \item $\mathcal{Q}\to \mathcal{P}\lor \mathcal{Q}$
        \item $\mathcal{P}\land \mathcal{Q}\to \mathcal{P}$
        \item $\mathcal{P}\land \mathcal{Q}\to \mathcal{Q}$
    \end{itemize}
    \end{multicols}
    \end{center}
\end{customthm}
\end{document}
